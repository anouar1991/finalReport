\paragraph{}
This report presents the results of the studies we conducted to obtain the knowledge needed to achieve our goal of automatically calculating the tuberculosis severity score via computed tomography (CT) image analysis.
\paragraph{}
As a first step, we conducted a bibliographic study on tuberculosis to get a clear idea about this disease and its impact, and its diffrent types. Then we introduced the concept of deep learning and its architectures. Afterwards, we talked about the field of medical imaging, focusing on a few applications of deep learning in the field of medical imaging and the file formats used to represent a medical image. Finally, we explored the work that has been done in the international challenge ImageCLEF2018\cite{ImageCLEF:1} regarding the classification of tuberculosis severity scoring.
\paragraph{}
Through this bibliographic study, we have been able to make several conclusions. First, Tuberculosis disease continues to lead the list of mortal disease, so there is a need for ways to increase the rate of decline of the disease. Eventually, automated dignosis and scoring of tuberculosis severity woud increase the availability and the time of diagnosis. Secondly, deep learning has imposed its supremacy and proven its efficiency in the field of image analysis. It is therefore worth the eandeavor to accelerate and improve image analysis. Finally, the work that has been done by the top performing teams to score pulmonary tuberculosis severity as part of the international ImageCLEF challenge showed the results obtained were very promising and yet there is still room for improvement.
\paragraph{}
In the practical part of this project we will try to apply deep learning and use the medical deep learning framework niftyNet to explore various ways to avert the weak points of the work of the state of the art.Eventually, reaching better results. 
