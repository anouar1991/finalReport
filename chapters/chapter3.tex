\paragraph{}
The Severity score is a cumulative score of severity of \acs{tb} case assigned by a medical doctor. Originally, the score varied from 1 ("critical/very bad") to 5 ("very good"). In this subtask, the score value is simplified so that values 1, 2 and 3 correspond to "high severity" class, and values 4 and 5 correspond to "low severity". In the process of scoring, the medical doctors considered many factors like pattern of lesions, results of microbiological tests, duration of treatment, patient's age and some other.
\paragraph{}
This task was proposed by ImageCLEF 2019 which is an evaluation campaign that is being organized as part of the CLEF initiative labs. \acs{svr} task lays under the ImageCLEFmed Tuberculosis category of tasks along with \acs{ct} report task (\acs{ctr}) which consists of generating an automatic report based on the \acs{ct} image of a patient.

This task possesses different challenges. The small amount of dataset and examples for the use of deep learning, the large dimensions of the image slices without the availability of an expert to assist in reducing the irrelevant slices and decreasing the dimensions. The lack of powerful computation resources makes it difficult to train a machine learning model,papticularly a deep neural network.

In the next sections we will describe the dataset being used for ImageCLEF 2019 \acs{svr} task and discuss the participations and the results obtained for the severity scoring of ImageCLEF 2018.
\subsection{Dataset}
The dataset contains 335 chest 3D CT images with an image size per slice of 512*512 pixels and number of slices varying from about 50 to 400 of TB patients. All the CT images are stored in NIFTI file format with .nii.gz file extension (g-zipped .nii files). In addition to the CT images, a set of clinically relevant metadata was given. The selected metadata includes the following binary measures: disability, relapse, symptoms of TB, comorbidity, bacillary, drug resistance, higher education, ex-prisoner, alcoholic, smoking, severity. Moreover, automatically extracted lung masks where provided along with a set of tools for loading/saving niftii files. 218 patients are used for training and 117 for test.
\subsection{Participations and results}
\paragraph{}
In 2018 there were 7 teams that participated and submitted their result in the \acs{svr} task.Each team used its own approach to build a solution for the SVR task. the participants were allowed to ubmit up to 10 runs for the task. A run consists of a submission of the results of the developped model when applied on a test set provided by imageCLEF.The runs submitted for the severity scoring subtask were evaluated in two ways. One used the original severity scores from 1 to 5 and the task was to predict those numerical scores as precise as possible (a regression problem). Here, Root Mean Square Error (RMSE) was computed between the ground truth severity and the predicted scores provided by the participants. Alternatively, the original severity score was transformed into two classes, where scores from 1 to 3 corresponded to ”high severity” and the 4 and 5 scores corresponded to the ”low severity” class. In this case the participants had to provide the probability of TB cases to belong to the ”high severity” class. The corresponding results were evaluated using AUC. in this section we present the 3 top performing approaches in terms of \ac{rmse} along with the approach used by MostaganemFSEI team.
\paragraph{}
Top \acs{rmse} achieved was 0.7840, by the team UIIP\_BioMed with a single run.A Coder-Decoder Convolutional Neural Network trained on a third-party dataset of 149 CT scans with lesions labeled by a qualified radiologist was used as a Lesion-based TB-descriptor extraction method. The lesion-based TB-descriptors extracted from the patients \acs{ct} scans were used to generate a random forest of a 100 trees as a classifier[--------------].This appraoch had the best \acs{rmse} 0.7840 and an AUC of 0.7708. The second best \acs{rmse} was 0.8513, achieved by the team MedGIFT among 9 runs. A graph model of the lung based on regional 3D texture features was used in an \acs{svm} classifier [-----------]. This method resulted in a better \acs{auc} than the previous approach.
