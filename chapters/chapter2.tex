\section{Medical imaging}
Medical imaging encompasses the technologies and processes of creating visual representations out of an anatomical volume in a form of images to be used for clinical diagnosis, medical intervention, and disease monitoring.
\subsection{Medical imaging technologies}
Comon medical imaging technologies include ones which use electromagnitic radiations such as X-ray imaging and Computed Tumography (\ac{ct}) imaging,others use sound waves and magnetic field such as ultrasound an magnetic resonance imaging.
\paragraph{X-ray imaging} this technology is the oldest and the most frequently used, it works on wave lengths and frequencies that can penetrate through the skin creating a visualisation of the inner body. It used to detect skeletal system malfunctioning, cancer through mamography, and other diagnosis that involves the visualisation of the inner body. This technique comes with risks associated with the use of X-ray radiation.
\paragraph{\ac{ct} imaging} it is a form of X-ray imaging that produces \ac{3d} visualisation of for diagnosis, providing greater quality and detailed imaging of the internal organs, bones, blood vessels, soft tissues within the body.
it also inherits the risk of X-ray imaging whereas the benefits exceeds its risk where in many cases the use of \ac{ct} scans prevents the need for exploratory surgery.
\paragraph{Ultrasound imaging} uses High-frequency sound waves that are transmitted from the probe to the body via the conducting gel, those waves then bounce back when they hit the different structures within the body and that is used to create an image for diagnosis.
This technology is considered the safes without any recorded side effects of its usage and is the most cost-effective.Due to its low risk, it is the first choice for pregnancy.
\paragraph{Magnetic resonance imaging} uses a strong magnetic field and radio waves it enables an indepth view of the inside of a joint or ligament to be seen, rather than just the outside as in the case of \ac{ct} scans and X-ray. It has risks associated with the use of strong magnetic field where any kind of metal implant, artificial joint could be moved or heated up whithin the magnetic field.
\section{Medical imaging Archiving and Recording} 
\paragraph{}
due to the nature of a medical image, storing it is diffrent from storing regular images.medical image data set consists typically of one or more images representing the projection of an anatomical volume onto an image plane (projection or planar imaging), a series of images representing thin slices through a volume (tomographic or multislice two-dimensional imaging), a set of data from a volume (volume or three-dimensional imaging), or multiple acquisition of the same tomographic or volume image over time to produce a dynamic series of acquisitions (four-dimensional imaging).\cite{ME:1}
\paragraph{}
There exist several medical images file formats all of them sharing the goal of standardizing medical images storage and transmission.Major file formats widely used in medical imaging are Analyze, Neuroimaging Informatics Technology Initiative (Nifti), Minc, and Digital Imaging and Communications in Medicine (Dicom).
\paragraph{Analyze}
Analysis 7.5 was created in the late 1980s as a format used by the Analyze commercial software developed at the Mayo Clinic in Rochester, MN, USA. For more than a decade, the format was the standard for post-processing medical imaging. The big point of view of the Analyze format is that it was designed for multidimensional (volume) data. Indeed, it is possible to store 3D or 4D data in a file (the fourth dimension being typically the temporal information). An Analyze 7.5 volume includes two binary files: an image file with the extension .img which contains the raw voxel data and a header file with the extension .hdr which contains the metadata (number of pixels in the three dimensions, voxel size and data type). The header has a fixed size of 348 bytes and is writen as a structure in C programming language. Reading and editing the header requires a utility software. The format is now considered "old" but it is still widely used and supported by many processing software, viewers, devices and conversion utilities.\cite{ME:1}
\paragraph{\ac{nifti}}
Nifti is a file format created in the early 2000s with the intention to create a format that preserves compatibility of the Analyze format but solving its weaknesses. Nifti may be considered a revised 'Analyze' format. NIfTI uses the "empty space" in the ANALYZE 7.5 header to add several new features such as image orientation with the intention of avoiding left-right ambiguity in the brain study. In addition, Nifti includes unsupported data type in the Analyze format as an unsigned 16-bit format. Although the format also allows the storage of header and pixel data in separate files, the images are usually saved as a single '.nii' file in which the header and data are stored. pixels are merged. The header has a size of 348 bytes in the case of data storage '.hdr' and '.img' and a size of 352 bytes in the case of a single file '.nii'. This difference in size is due to the presence of four additional bytes at the end, essentially to make the size a multiple of 16, and also to provide space for storing additional metadata.\cite{NIF:1,ME:1}
\paragraph{\ac{minc}}
The Minc file format was developed in 1992 to provide a flexible data format for medical imaging. The first version of the Minc format (Minc1) was based on the standard common network format (NetCDF). Subsequently, to overcome the large data file support constraint and provide new features, Minc's development team chose to upgrade from NetCDF to Hierarchical Data Format version 5 (HDF5). This new version which is not compatible with the previous one, was called Minc2.\cite{MIN:1,ME:1}
\paragraph{\ac{dicom}}
\ac{dicom} (Digital Imaging and Communications in Medicine), is the international standard for transmitting, storing, retrieving, printing, processing and displaying medical imaging information.Dicom can only store pixel values ​​as an integer.However, it supports various types of data, including floats, to store metadata. Whenever the values ​​stored in each voxel are to be scaled, Dicom uses a scaling factor using two fields in the header defining the slope and the intercept of the linear transformation to be converted in real values.Dicom supports compressed image data through a mechanism that encapsulates a non-Dicom document into a Dicom file. The compression systems supported by Dicom are \ac{jpeg}, Run-Length Encoding (\ac{rle}), JPEG-LS, JPEG-2000 and \ac{mpeg}2 / MPEG4.\cite{DIC:1,ME:1}\\
Table \ref{me-file-format} shows a summary of file formats characteristics
\begin{table}[h]
\begin{center}
\begin{tabular}{l p{5cm} l p{5cm}}
\hline
\textbf{Format} & \textbf{Header}                                                                              & \textbf{Extension} & \textbf{Data types}                                                                                     \\ \hline
Analyze & Fixed-length: 348 byte binary format                                                         & .img and .hdr & Unsigned integer (8-bit), signed integer (16-, 32-bit), float (32-, 64-bit), complex (64-bit)               \\ \hline
Nifti   & Fixed-length: 352 byte binary formata (348 byte in the case of data stored as .img and .hdr) & .nii          & Signed and unsigned integer (from 8- to 64-bit), float (from 32- to 128-bit), complex (from 64- to 256-bit) \\ \hline
Minc    & Extensible binary format                                                                     & .mnc          & Signed and unsigned integer (from 8- to 32-bit), float (32-, 64-bit), complex (32-, 64-bit)            
\end{tabular}
\caption{Summary of file formats characteristics \cite{ME:1}}
\label{me-file-format}
\end{center}
\end{table}