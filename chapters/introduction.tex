\paragraph{}
In recent decades, medical imaging has become indispensable in the diagnosis and therapy of diseases. With the enhancement of medical imaging databases, new methods are required to better handle this huge volume of data. However, because of the large variations and complexity of medical imaging data, it is generally difficult to deduce analytical solutions or simple methods to describe and represent objects such as lesions and anatomies in data. Therefore, medical imaging tasks require learning from the examples, and this is one of the key interests of the machine learning field.
\paragraph{}
Machine learning has become one of the main tools for medical image analysis. Machine learning techniques are solutions for developing tools to help physicians diagnose, predict, and prevent the risk of disease before it becomes too late in less time. Deep Learning is a new component in the field of machine learning that encompasses a wide range of network architectures designed to perform multiple tasks. The first use of neural networks for medical image analysis goes back more than twenty years, their use has increased by several orders of magnitude over the last five years. different, articles \cite{NNMEEX:1,NNMEEX:2,NNMEEX:3,NNMEEX:4,NNMEEX:5} have highlighted the application of deep learning to a wide range of medical image analysis tasks (segmentation, classification, detection, recording, image reconstruction, enhancement, etc\dots).
\paragraph{}
Tuberculosis is an infectious disease caused by a bacterium called Bacillus mycobacterium tuberculosis \cite{TBT:1}. In 2018, 10 million people fell ill with \ac{tb}, and 1. 6 million died from the disease \cite{TBT:1}. This disease remained one of the top ten leading causes of death in the world in 2018 \cite{TBT:1}. Tuberculosis attacks the lungs but can also affect other parts of the body \cite{TBT:2}. Accurate and rapid diagnosis is the key to controlling this disease, but traditional \ac{tb} tests produce inaccurate or time consumnig results to be definitive. Researchers have been interested in this disease, particularly in the context of the ImageCLEF 2018\cite{ImageCLEF:1} international challenge \cite{ImageCLEF:1} where two tasks have been reserved for it. Algorithms involving deep learning have been tested to diagnose the presence or absence of tuberculosis. The results obtained were interesting. Indeed, the algorithms have achieved an impressive accuracy rate up to 96\% \cite{NNMEEX:6,NNMEEX:7} a result that is better than the intervention of many radiologists.
\paragraph{}
The goal of our project is to automatically give score of \ac{tb} severity via \ac{ct} scan. One of the possible applications of this study is to accelerate the diagnosis of the disease from a radiology image without resorting to expensive medical tests. This work is part of the ImageCLEF2018  task of classifying types of \ac{tb}, which has shown more promising results. We explore in this paper the different work and different concepts that link with this problem. Starting by giving an overview of tuberculosis and its types in chapter 1. Then, discuss the relationship between artificial intelligence and medical images then give definition of some important parts of these fields in chapter 2. In chapter 3, the ImageCLEF Tasks are described and the related work of tuberculosis severity scoring is reviewed.