\paragraph{}
In recent decades, medical imaging has become indispensable in the diagnosis and therapy of diseases. With the enhancement of medical imaging databases, new methods are required to better handle this huge volume of data. However, because of the large variations and complexity of medical imaging data, it is generally difficult to deduce analytical solutions or simple methods to describe and represent objects such as lesions and anatomies in data. Therefore, medical imaging tasks require learning from the examples, and this is one of the key interests of the machine learning field.
\paragraph{}
Machine learning has become one of the main tools for medical image analysis. Machine learning techniques are solutions for developing tools to help physicians diagnose, predict, and prevent the risk of disease before it becomes too late. Deep Learning is a new component in the field of machine learning that encompasses a wide range of network architectures designed to perform multiple tasks. The first use of neural networks for medical image analysis goes back more than twenty years [1], their use has increased by several orders of magnitude over the last five years. Recent reviews [2] have highlighted the application of deep learning to a wide range of medical image analysis tasks (segmentation, classification, detection, recording, image reconstruction, enhancement, etc.).
\paragraph{}
Tuberculosis is an infectious disease caused by a bacterium called Bacillus mycobacterium-tuberculosis. With an estimated 10.4 million new TB cases and a global mortality rate of 1.8 million, this disease remained one of the top ten leading causes of death in the world in 2015. Tuberculosis attacks the lungs but can also affect other parts of the body. Accurate and rapid diagnosis is the key to controlling this disease, but traditional TB tests produce inaccurate or too long results to be definitive. Researchers have been interested in this disease, particularly in the context of the ImageCLEF 2017 international challenge [3] where two tasks have been reserved for it. Algorithms involving deep learning have been tested to diagnose the presence or absence of tuberculosis. The results obtained were interesting. Indeed, the algorithms have achieved an impressive accuracy rate of 96\% [4] a result that is better than the intervention of many radiologists.
\paragraph{}
The goal of our project is to automatically detect TB types via CT scan. One of the possible applications of this study is to accelerate the diagnosis of the disease from a radiology image without resorting to expensive medical tests. This work is part of the ImageCLEF2017 [3] task of classifying types of TB, which has shown more promising results. We explore in this paper the different work and different concepts that link with this problem.